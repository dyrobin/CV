
% Select 'print' for paper-printing version
\documentclass[print]{mcdowellcv}

% For mathematical symbols
\usepackage{amsmath}

% Disable hyphenation
\usepackage[none]{hyphenat}

% For GitHub and LinkedIn
\usepackage{fontawesome}
\usepackage[hidelinks,baseurl=https://]{hyperref}

\name{Yang Deng}
\address{\mbox{\faGithub\quad\url{github.com/dyrobin}} \linebreak \mbox{\faLinkedin\quad\url{linkedin.com/in/dyrobin}}}
\contacts{+358 (0) 50 5628551 \linebreak \url{yang.r.deng@gmail.com}}

\begin{document}

	% Print the header
	\makeheader
	
	% Print the content
	\begin{cvsection}{Employment}
		\begin{cvsubsection}{Software Architect}{LUMO Connections}{Nov. 2016 -- Apr. 2017}
			Lamp for Long Distance Relationships
			\begin{itemize}
				\item Built up a Node.js communication system including a client on Raspberry Pi and a server on Amazon EC2
				\item Responsible for design and implementation of communication module that integrates with WiFi manager and 
				sends real-time messages via WebSocket on ESP8266 (NodeMCU)
			\end{itemize}
		\end{cvsubsection}
			
		\begin{cvsubsection}{Software Engineer, Researcher}{Aalto University}{Sept. 2011 -- Oct. 2016}
			Adaptive 6LoWPAN Packet Size Control (2014 - 2016)
			\begin{itemize}
				\item Developed a SciPy tool to process massive measurement data and analyze 6LoWPAN fragmentation issue
				\item Designed and implemented a \textit{first linear then exponential} packet size control mechanism that improves throughput by 40\% in best case
				\item Published and presented a paper in EWSN2015 (best conference on wireless sensor networks in Europe)
			\end{itemize}
			\smallskip
			Multicast for Constrained Networks (2013 - 2015)
			\begin{itemize}
				\item Implemented a simplified IPv6 Multicast Listener Discovery on Contiki OS
				\item Established a testbed for MPL (RFC7731) and evaluated its performance against flooding 
				\item Designed and implemented a simple multicast protocol for constrained networks based on PIM-SSM
			\end{itemize}
			\smallskip
			MAMMoTH - Massive Scale Machine to Machine Service (2011 - 2013)
			\begin{itemize}
				\item Implemented a basic HTTP-CoAP mapping layer on Contiki OS
				\item Deployed around 1000 emulated constrained devices to verify the scalability of the platform
			\end{itemize}
		\end{cvsubsection}

		\begin{cvsubsection}{Software Engineer}{Nokia Research Center, China}{Nov. 2008 -- Aug. 2011}
			Rich Context Mobile Platform (2010 - 2011)
			\begin{itemize}
				\item Designed and implemented a publish-subscribe messaging mechanism and a QT GUI on N900 device
				\item Filed a patent (US 20140237486 A1) on inter-widget communication
			\end{itemize}
			\smallskip
			Handwriting Calculator (2009 - 2010)
			\begin{itemize}
				\item Developed GTK+ GUI (including formula display and stylus gestures) on Maemo platform
				\item Published on Nokia Beta Labs and won outstanding project award
			\end{itemize}
		\end{cvsubsection}

		\begin{cvsubsection}{Software Tester, Intern}{IBM Corporation, China}{June 2008 -- Aug. 2008}			
			\begin{itemize}
				\item Localization and globalization testing for IBM Lotus software
				\item Developed a Perl tool to detect empty translations of the content
			\end{itemize}
		\end{cvsubsection}
	\end{cvsection}

	\begin{cvsection}{Additional Experience}
		\begin{cvsubsection}{}{}{}	
			\begin{itemize}
				\item Web Developer (2016): Built an e-commerce website (\url{store.kanssani.fi}) by WooCommerce
				\item Website Administrator(2012 - 2016): Maintained a forum for \href{https://cssa.ayy.fi}{CSSA-Espoo}
				(once largest Chinese student community in Greater Helsinki area)
				\item Teaching Assistant (2015): Taught two course assignments (P2P networking and WiFi indoor positioning)
			\end{itemize}
		\end{cvsubsection}
	\end{cvsection}
	
	\begin{cvsection}{Education}
		\begin{cvsubsection}{Espoo, Finland}{Aalto University}{Sept. 2011 -- Aug. 2016}
			\begin{itemize}
				\item Licentiate degree candidate in Computer Science and Engineering. GPA: 4.77/5
			\end{itemize}
		\end{cvsubsection}

		\begin{cvsubsection}[2]{Beijing, China}{Beijing University of Posts and Communications}{Sept. 2003 --  Apr. 2010}
			\begin{itemize}
				\item M.S.E in Computer Networking, April 2010. GPA: 4.0/5				 
				\item B.S.E in Computer Science and Engineering, June 2007. GPA: 4.4/5
			\end{itemize}
		\end{cvsubsection}
	\end{cvsection}
	
	\begin{cvsection}{Skills}
		\begin{cvsubsection}{}{}{}	
			\begin{itemize}
				\item Languages: C; C++; Java; Unix Shell; Python; PHP; JavaScript; XML; HTML/CSS; SQL
				\item Platform: Unix/Linux; Contiki/Arduino embedded devices
				\item Networking: 802.15.4; 6LoWPAN; TCP/IP; HTTP; CoAP; RESTful API; Apache
				\item Tools: Makefile; Git
			\end{itemize}
		\end{cvsubsection}
	\end{cvsection}

\end{document}

I built a small full IP network to make different radio-type devices communicate directly. The scenario worked like this: if a Zigbee motion detector finds people moving then another Zigbee light is switched on and a WiFi camera starts to work. The scenario itself is simple but the amazing part is that no gateway was used to connect them. Instead, every device had its own auto-configured IP address and was able to be controlled by any device on the Internet, for example, a laptop, a cell phone or even an application server. The controlling protocol is CoAP, an HTTP-like protocol specially designed for future IoT. Comparing to traditional gateway, full IP network will decrease the system complexity greatly as it doesn’t need to maintain different stacks for different radios. This prototyped network is actually an extension of 6LoWPAN and as Bluetooth SIG declare that IP will be supported in their future I believe full IP network will become the mainstream connectivity solution.
