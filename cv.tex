%% The MIT License (MIT)
%%
%% Copyright (c) 2015 Daniil Belyakov
%%
%% Permission is hereby granted, free of charge, to any person obtaining a copy
%% of this software and associated documentation files (the "Software"), to deal
%% in the Software without restriction, including without limitation the rights
%% to use, copy, modify, merge, publish, distribute, sublicense, and/or sell
%% copies of the Software, and to permit persons to whom the Software is
%% furnished to do so, subject to the following conditions:
%%
%% The above copyright notice and this permission notice shall be included in all
%% copies or substantial portions of the Software.
%%
%% THE SOFTWARE IS PROVIDED "AS IS", WITHOUT WARRANTY OF ANY KIND, EXPRESS OR
%% IMPLIED, INCLUDING BUT NOT LIMITED TO THE WARRANTIES OF MERCHANTABILITY,
%% FITNESS FOR A PARTICULAR PURPOSE AND NONINFRINGEMENT. IN NO EVENT SHALL THE
%% AUTHORS OR COPYRIGHT HOLDERS BE LIABLE FOR ANY CLAIM, DAMAGES OR OTHER
%% LIABILITY, WHETHER IN AN ACTION OF CONTRACT, TORT OR OTHERWISE, ARISING FROM,
%% OUT OF OR IN CONNECTION WITH THE SOFTWARE OR THE USE OR OTHER DEALINGS IN THE
%% SOFTWARE.

% The font could be set to Windows-specific Calibri by using the 'calibri' option
\documentclass[]{mcdowellcv}

% For mathematical symbols
\usepackage{amsmath}

% Set applicant's personal data for header
\name{YANG \hspace{2mm} DENG}
\address{}
\contacts{+358 (0) 50 5628551 \linebreak yang@lumo.co.com}

\begin{document}

	% Print the header
	\makeheader
	
	% Print the content
	\begin{cvsection}{Employment}
		\begin{cvsubsection}{Software Engineer, Researcher}{Aalto University}{September 2011 -- October 2016}
			Adaptive Packet Size Control for 6LoWPAN (2014 - 2016)
			\begin{itemize}
				\item Analyzed fragment issue of data transmission in 6LoWPAN
				\item Designed a \textquotedblleft linear first then exponential\textquotedblright \space algorithm to improve the throughput by 40\% at best
				\item Published a paper on EWSN (best sensor network conference in Europe, also famous in the world)
			\end{itemize}
			Multicast for Constrained Networks (2013 - 2015)
			\begin{itemize}
				\item Implemented simplified MLD on Contiki OS
				\item Evaluated performance of MPL and flooding in terms of different metrics
				\item Developed a simple multicast protocol for constrained network based on PIM-SSM
			\end{itemize}
			MAMMoTH - Massive Scale Machine to Machine Service (2011 - 2013)
			\begin{itemize}
				\item Implemented an HTTP-CoAP mapping layer on Contiki OS
				\item Deployed more than 1000 emulated and real constrained devices on the platform
			\end{itemize}
		\end{cvsubsection}
		
		\begin{cvsubsection}{Web Administrator}{CSSA-Espoo ry}{January 2012 -- December 2015}
			\begin{itemize}
				\item Maintained a phpBB forum that provides information for all Chinese students in Greater Helsinki
				\item Created a customized WordPress website for the organization
			\end{itemize}
		\end{cvsubsection}

		\begin{cvsubsection}{Software Engineer}{Nokia Research Center, China}{May 2010 -- August 2011}
			Rich Context Mobile Platform			
			\begin{itemize}
				\item Designed and implemented a publish-subscribe messaging mechanism and a QT GUI on N900 device
				\item Filed a patent application on inter-widget communication
			\end{itemize}
		\end{cvsubsection}

		\begin{cvsubsection}{Software Tester, Intern}{IBM Corporation, China}{June 2009 -- September 2009}
			\begin{itemize}
				\item Localization and globalization testing for Lotus Software
				\item Wrote a PERL script to detect empty translations of the content
			\end{itemize}
		\end{cvsubsection}
				
		\begin{cvsubsection}{Software Engineer, Intern}{Nokia Corporation, China}{Noverber 2007 -- December 2008}
			Handwriting Calculator
			\begin{itemize}
				\item Developed GTK+ GUI (including formula display layout and stylus handy gestures) on Maemo platform
				\item Won outstanding project award issued by Nokia Research Center 
			\end{itemize}
		\end{cvsubsection}
	\end{cvsection}
	
	\begin{cvsection}{Education}
		\begin{cvsubsection}{Espoo, Finland}{Aalto University}{September 2011 -- May 2016}
			\begin{itemize}
				\item Licentiate candidate in Computer Science and Engineering. GPA: 4.77/5 \newline
				Thesis: Enhancement of 6LoWPAN Transmission in Lossy and Low Power Networks
			\end{itemize}
		\end{cvsubsection}

		\begin{cvsubsection}{Beijing, China}{Beijing University of Posts and Communications}{September 2003 --  April 2010}
			\vspace{0.2cm}
			\begin{itemize}
				\item M.S.E in Computer Networking. GPA: 4.0/5 \newline
				 Thesis: Research on Architecture of Telecom Mashup Platform. Grade: outstanding
				\item B.S.E in Computer Science and Engineering. GPA: 4.4/5 \newline
				Thesis: Client Implementation in Video Monitoring System. Grade: outstanding
			\end{itemize}
		\end{cvsubsection}
	\end{cvsection}
	
	\begin{cvsection}{Additional Experience}
		\begin{cvsubsection}{}{}{}	
			\begin{itemize}
				\item Web Developer (2016): Built an e-commerce website by WooCommerce for Kanssani
				\item Teaching Assistant (2015): Taught two assignments (P2P networking and indoor position by WIFI fingerprint)
				\item Thesis Instructor (2013 - 2014): Instructed two master students' theses on sensor networking
			\end{itemize}
		\end{cvsubsection}
	\end{cvsection}
	
%	\begin{cvsection}{Languages and Technologies}
%		\begin{cvsubsection}{}{}{}	
%			\begin{itemize}
%				\item C++; C; Java; Objective-C; C\#.NET; SQL; JavaScript; XSLT; XML (XSD) Schema 
%				\item Visual Studio; Microsoft SQL Server; Eclipse; XCode; Interface Builder
%			\end{itemize}
%		\end{cvsubsection}
%	\end{cvsection}

\end{document}

I built a small full IP network to make different radio-type devices communicate directly. The scenario worked like this: if a Zigbee motion detector finds people moving then another Zigbee light is switched on and a WiFi camera starts to work. The scenario itself is simple but the amazing part is that no gateway was used to connect them. Instead, every device had its own auto-configured IP address and was able to be controlled by any device on the Internet, for example, a laptop, a cell phone or even an application server. The controlling protocol is CoAP, an HTTP-like protocol specially designed for future IoT. Comparing to traditional gateway, full IP network will decrease the system complexity greatly as it doesn’t need to maintain different stacks for different radios. This prototyped network is actually an extension of 6LoWPAN and as Bluetooth SIG declare that IP will be supported in their future I believe full IP network will become the mainstream connectivity solution.